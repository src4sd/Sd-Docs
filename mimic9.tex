\documentclass[12pt]{article}
\usepackage{tex-squares,multicol}
\usepackage{sd}
\usepackage{sd-pf}
\usepackage{sd-sf}

\begin{document}

\Largedancer

\title{The Mimic Concept}
\author{\cbox{Sue Curtis \\ Sharon, MA}\hspace{1in}\cbox{Bill Ackerman\\N. Billerica, MA}}
\date{December 2007 \\ Last Updated May 22, 2023}
\maketitle

\parskip 12pt
\section{Introduction}

The \emph{Mimic} concept is a way of getting everyone to do a designated
part of the call.
For example, \emph{Mimic Leads} means that everyone does the leads'
part of the call.
\emph{Mimic Beaus} means that everyone does the beaus' part of the call.
\emph{Mimic Centers} is generally the same as Central in cases where
Central is proper, but \emph{Mimic Centers} applies to more calls.
\emph{Mimic Centers} and \emph{Mimic Ends} are described in more detail
in a separate paper\footnote{Sue Curtis, ``The Mimic Concept: Centers and Ends'', December 2009.};
the present paper focuses on cases involving beaus/belles and leaders/trailers.

The definition of \emph{Mimic}, given below, involves making an adjustment,
doing the call, and then undoing the adjustment.
We expect that dancers will eventually learn to do many of the
\emph{Mimic} calls in a single smooth flowing motion, without
making any adjustments, as they now do Central.
However, when dancers are first learning \emph{Mimic}, or when they
later encounter new or unusual examples, they may wish to physically
make the adjustment.  Thus, in this paper, we will describe the
calls in terms of making an adjustment.
You can think of this as similar to the parallelogram adjustment:
we explain the call in terms of an adjustment, but we only physically
make the adjustment on harder calls.

The next section gives the definition of the concept
and a number of examples, starting with simple two-person
calls and progressing to eight-person calls.
Section 3 discusses the syntax in more detail and
illustrates when it is essential to explicitly name the setup.
Section 4 introduces some alternative designators.
Section 5 concludes the paper and includes some thoughts on how
this concept relates to other similar concepts.

\section{Definition and Examples}

To do Mimic calls, first replace each dancer with a pair of dancers,
so that the real dancer takes the position designated.
For example, if the call is \emph{Mimic Leads},
each dancer effectively places a phantom behind them,
so that they become leads in their pair.
Second, do the call in the newly formed phantom setup.
Finally, squeeze out (remove) the added phantoms and collapse
the real people into a compact setup.
Often, there is only one way to remove the phantoms and
collapse the remaining setup.
If there are multiple ways to do this, 
\emph{remove the phantoms so that the ending setup
has as nearly as possible the same shape and elongation axis
as the original setup.}

The complete syntax of the \emph{Mimic} concept is
\emph{Mimic $\langle$designator$\rangle$ of $\langle$setup$\rangle$}.
For example, the caller might say \emph{Mimic Leads of Lines}
or \emph{Mimic Trailers of Columns}.
The setup given specifies the setup in which the dancers do the call
as well as phantoms that can be used for collapsing at the end.
For example, on \emph{Mimic Leads of Lines}, dancers add phantoms
behind themselves, do the call in a newly-formed 2x4 line setup,
and then collapse with the phantoms from that setup.
The shorter syntax (e.g. \emph{Mimic Leads}), is permissible
when there is only one possible starting setup for the call.

The facing direction of the added phantoms is not automatically
specified by the \emph{Mimic} concept, but may be specified by
the setup named.  For example, on \emph{Mimic Leads of Waves},
the facing direction of the phantoms can be assumed to make waves,
but on \emph{Mimic Leads of Lines} or \emph{Mimic Leads of Columns},
the phantoms' facing directions are not specified.
This is no different from many phantom concepts, 
such as Split Phantom Boxes, where the facing direction is not specified.
If the caller does not name a setup that specifies facing directions,
the call must be one that is unambiguous when
working with phantoms of unknown facing directions.

Note that the \emph{Mimic} adjustment always doubles the size of the setup.
Thus, \emph{Mimic} calls generally require half the number of people
as the base call.  The \emph{Mimic} version of a two-person call
will generally be a one-person call, and the \emph{Mimic} version
of a four-person call will generally be a two-person call.

First, let's consider \emph{Mimic} versions of some two-person calls
and then gradually build up to some larger setups.
The two-person calls are often the easiest, because
the \emph{Mimic} versions are one-person calls.
With a one-person call, you always end back on the spot you
started on.  How far off can you be, when you know you have
to end on the same spot you started on?
In addition, the shorter syntax, such as \emph{Mimic Leads}
or \emph{Mimic Beaus}, is generally sufficient, since you will
be unambiguously working in the setup that consists of you and your phantom.

Below are some examples of applying \emph{Mimic} to two-person calls.

\example{Mimic Beaus, Shazam}
\begin{displaydance}
\phone{b1n} & \phtwo{b1n,p..} & \pvtwo{b1w,p..} & \phone{b1w} \cr
\cbox{Before Call} & \cbox{After Adding Phantom} & \cbox{After Shazam} & \cbox{Finished}\\
\end{displaydance}
\endexample

\example{Mimic Leads, Latch On}
\begin{displaydance}
\phone{b1n} & \pvtwo{b1n,p..} & \pvtwo{p..,b1w} & \phone{b1w} \cr
\cbox{Before Call} & \cbox{After Adding Phantom} & \cbox{After Latch On} & \cbox{Finished}\\
\end{displaydance}
\endexample

\example{Mimic Leads, Single Shakedown}
\begin{displaydance}
\phone{b1n} & \pvtwo{b1n,p..} & \phtwo{p..,b1w} & \phone{b1w} \cr
\cbox{Before Call} & \cbox{After Adding Phantom} & \cbox{After Single Shakedown} & \cbox{Finished}\\
\end{displaydance}
\endexample

All of these are just turn 3/4 to the right.

Being one-person calls, they can be done anywhere:

\example{Girls, Mimic Beaus, Shazam}
\begin{displaydance}
\phhourglass{b1n,b2e,g1s,g2s,b3s,b4w,g3n,g4n} & \phhourglass{b1n,b2e,g1e,g2e,b3s,b4w,g3w,g4w} \cr
\cbox{Before} & \cbox{After}\\
\end{displaydance}
\endexample

When the given call is a four-person call,
the \emph{Mimic} version is generally a two-person call.
The shorter syntax is again usually sufficient,
since most four-person calls are either clearly box calls
or clearly line calls.  
The removal of phantoms at the end may be straightforward
or may be slightly more complicated.
If the call finishes in a 1x4, removal of the phantoms is always
straightforward: just remove the phantoms wherever they are and have
the real people slide together.
If the call finishes in a 2x2 with the real people not in opposite corners,
the removal of phantoms is again straightforward: 
just remove the phantoms and collapse the 2x2 to a 1x2.
If the real people are in opposite corners, move them as necessary
to give a 1x2 result that has the same elongation axis as their starting 1x2.

Below are some examples, starting with cases where there is only one
way to remove the phantoms.

\example{Mimic Leads, Shakedown}
\begin{displaydance}
\phtwo{b1n,b2n} & \phBB{b1n,p..,b2n,p..} & \phBB{p..,p..,b1w,b2w} & \pvtwo{b1w,b2w} \cr
\cbox{Before Call} & \cbox{After Adding Phantoms} & \cbox{After Shakedown} & \cbox{Finished}\\
\end{displaydance}
\endexample

When danced smoothly, this is As Couples 1/4 Right and
(individually) Roll twice.
It feels like Shakedown, but starts and ends in a 1x2.

\example{Mimic Leads, Wheel the Ocean}
\begin{displaydance}
\phtwo{b1n,b2n} & \phBB{b1n,p..,b2n,p..} & \phBB{p..,p..,b1s,b2s} & \pvtwo{b1s,b2s} \cr
\cbox{Before Call} & \cbox{After Adding Phantoms} & \cbox{After Wheel the Ocean} & \cbox{Finished}\\
\end{displaydance}
\endexample

When danced smoothly, this is Wheel Around and put the belle in
front of the beau (1/2 Half Sashay).
In fact,
\emph{Mimic Leads, Wheel the Sea} has exactly the same result!
You may think of it as stepping to left hands with the phantoms
instead of right hands.
But when the phantoms are removed, it doesn't matter which hand
you had with them.

\example{Mimic Beaus, Cross and Turn}
\begin{displaydance}
\pvtwo{b2s,b1n} & \phBB{p..,b1n,b2s,p..} & \phBB{p..,b2s,b1n,p..} & \pvtwo{b1n,b2s} \cr
\cbox{Before Call} & \cbox{After Adding Phantoms} & \cbox{After Cross and Turn} & \cbox{Finished}\\
\end{displaydance}
\endexample

In the example above, there were two choices for the final adjustment:
step back or slide to the left.  The latter is the correct
one because it preserves the elongation axis of the original 1x2.
Below is another example of this type.

\example{Mimic Leads, Stack the Line}
\displayfour{\dancer{1}{n}\dancer{2}{n}}{Before Call}
            {\dancer{1}{n}\dancer{2}{n}\cr
             \pdancer{}{x}\pdancer{}{x}}{After Adding Phantoms}
	    {\pdancer{}{x}\dancer{1}{e}\cr
             \dancer{2}{w}\pdancer{}{x}}{After Stack the Line}
	    {\dancer{2}{w}\dancer{1}{e}}{After Entire Call}
\endexample

\emph{Mimic Beaus} can be used from couples or from tandem dancers.
Below are a few examples.

\example{Mimic Beaus, Swing Thru}
\begin{displaydance}
\phtwo{b1n,b2n} & \phfour{b1n,p..,b2n,p..} & \phfour{p..,p..,b1n,b2s} & \phtwo{b1n,b2s} \cr
\cbox{Before Call} & \cbox{After Adding Phantoms} & \cbox{After Swing Thru} & \cbox{Finished}\\
\end{displaydance}
\endexample

\example{Mimic Beaus, Flip Back}
\begin{displaydance}
\phtwo{b1n,b2n} & \phfour{b1n,p..,b2n,p..} & \phBB{p..,p..,b2e,b1w} & \pvtwo{b2e,b1w} \cr
\cbox{Before Call} & \cbox{After Adding Phantoms} & \cbox{After Flip Back} & \cbox{Finished}\\
\end{displaydance}
\endexample

\example{Mimic Beaus, Quarter Thru}
\begin{displaydance}
\pvtwo{b1n,b2n} & \phBB{b1n,b2n,p..,p..} & \pvfour{b1e,b2w,p..,p..} & \pvtwo{b1e,b2w} \cr
\cbox{Before Call} & \cbox{After Adding Phantoms} & \cbox{After Quarter Thru} & \cbox{Finished}\\
\end{displaydance}
\endexample

\example{Mimic Beaus, Peel and Trail}
\begin{displaydance}
\pvtwo{b1n,b2n} & \phBB{b1n,b2n,p..,p..} & \phfour{b1s,p..,b2s,p..} & \phtwo{b1s,b2s} \cr
\cbox{Before Call} & \cbox{After Adding Phantoms} & \cbox{After Peel and Trail} & \cbox{Finished}\\
\end{displaydance}
\endexample

The \emph{Mimic} concept nests in the expected way with other concepts.
On a \emph{Concentric, Mimic Leads, Shakedown} from a tidal two-faced line,
the centers and ends do their \emph{Mimic Leads, Shakedown} separately,
and then, after squeezing out the phantoms, reassemble in the appropriate way.

\example{Concentric, Mimic Leads, Shakedown}
\begin{displaydance}
\pheight{b1s,b2s,b3n,b4n,b5s,b6s,b7n,b8n} & \pvBBBB{b7w,b1e,b5e,b3w,b6e,b4w,b8w,b2e} \cr
\cbox{Before} & \cbox{After}\\
\end{displaydance}
\endexample

When the base call is an eight-person call,
the \emph{Mimic} version is generally a four-person call.
While the shorter syntax is often unambiguous,
we find that dancers often find it helpful
to have the setup named explicitly.
In this paper, we will use the full syntax on most eight-person calls.
As with four-person calls,
the removal of phantoms at the end may be straightforward
(i.e. only one possibility) or may require explicitly
choosing the setup that is as close as possible in shape
and elongation to the original starting setup.

Below are a few examples where the ending should be straightforward.
In these examples, we will use the full syntax of the concept.
It might seem at first that many of these examples would not
require naming the setup.
However, bear in mind that in practice, there will be eight people doing
the call rather than just the four we show, and therefore it will
be harder to determine the setup in which the actual call is done.
This issue will be discussed in more detail in
Section \ref{specifyingsetups}.

\example{Mimic Leads of Lines, Link Up}
\begin{displaydance}
\phfour{b1n,b2n,b3s,b4s} & \phBBBB{b1n,p..,b2n,p..,p..,b3s,p..,b4s} & \phBBBB{b2s,b1s,p..,p..,p..,p..,b4n,b3n} & \phBB{b2s,b1s,b4n,b3n} \cr
\cbox{Before Call} & \cbox{After Adding Phantoms} & \cbox{After Link Up} & \cbox{Finished}\\
\end{displaydance}
\endexample

\example{Mimic Leads of Columns, Trade By}
\begin{displaydance}
\phBB{b1n,b3n,b2n,b4n} & \pvBBBB{b2n,b1n,p..,p..,b4n,b3n,p..,p..} & \pvBBBB{b1s,b2s,b4n,b3n,p..,p..,p..,p..} & \phBB{b2s,b3n,b1s,b4n} \cr
\cbox{Before Call} & \cbox{After Adding Phantoms} & \cbox{After Trade By} & \cbox{Finished}\\
\end{displaydance}
\endexample

% sue: from original paper, not sure it adds value here
%
% \example{Mimic Leads, Finish Perk Up}
% \displayfour
% {\dancer{1}{n}\dancer{2}{s}\dancer{3}{n}\dancer{4}{s}}{Before Call}
% {\dancer{1}{n}\pdancer{}{x}\dancer{3}{n}\pdancer{}{x}\cr
%  \pdancer{}{x}\dancer{2}{s}\pdancer{}{x}\dancer{4}{s}}
% {After Adding Phantoms}
% {\dancer{4}{e}\dancer{2}{e}\cr
%  \pdancer{}{x}\pdancer{}{x}\cr
%  \pdancer{}{x}\pdancer{}{x}\cr
%  \dancer{3}{w}\dancer{1}{w}\cr}
% {After Finish Perk Up}
% {\dancer{4}{e}\dancer{2}{e}\cr
%  \dancer{3}{w}\dancer{1}{w}\cr}
% {After Entire Call}
% \endexample

\example{Mimic Trailers of Columns, Wind the Bobbin}
\displayfour
{\dancer{1}{n}\dancer{2}{s}\cr
 \dancer{3}{n}\dancer{4}{s}}{Before Call}
{\pdancer{}{x}\dancer{2}{s}\cr
 \dancer{1}{n}\pdancer{}{x}\cr
 \pdancer{}{x}\dancer{4}{s}\cr
 \dancer{3}{n}\pdancer{}{x}}
{After Adding Phantoms}
{\pdancer{}{x}\dancer{2}{n}\dancer{4}{s}\pdancer{}{x}\cr
 \pdancer{}{x}\dancer{1}{n}\dancer{3}{s}\pdancer{}{x}\cr}
{After Wind the Bobbin}
{\dancer{2}{n}\dancer{4}{s}\cr
 \dancer{1}{n}\dancer{3}{s}\cr}
{After Entire Call}
\endexample

% sue: example from original paper, seems redundant here
% \example{Mimic Leads, Mini Busy}
% \displayfour
% {\dancer{1}{n}\dancer{2}{n}\dancer{3}{s}\dancer{4}{s}}
% {Before Call}
% {\dancer{1}{n}\dancer{2}{n}\pdancer{}{x}\pdancer{}{x}\cr
%  \pdancer{}{x}\pdancer{}{x}\dancer{3}{s}\dancer{4}{s}}
% {After Adding Phantoms}
% {\idancer\dancer{1}{s}\dancer{2}{s}\idancer\cr
% \pdancer{}{x}\pdancer{}{x}\pdancer{}{x}\pdancer{}{x}\cr
% \idancer\dancer{3}{n}\dancer{4}{n}\idancer}
% {After MiniBusy}
% {\dancer{1}{s}\dancer{2}{s}\cr
%  \dancer{3}{n}\dancer{4}{n}}
% {After Entire Call}
% \endexample

\example{Mimic Beaus of Columns, Track 2}
\begin{displaydance}
\phfour{b1w,b2w,b3e,b4e} & \phBBBB{p..,b1w,p..,b2w,b3e,p..,b4e,p..} & \pvBBBB{b1e,b2e,p..,p..,p..,p..,b3w,b4w} & \phBB{b2e,b4w,b1e,b3w} \cr
\cbox{Before Call} & \cbox{After Adding Phantoms} & \cbox{After Track 2} & \cbox{Finished}\\
\end{displaydance}
\endexample

\example{Mimic Beaus of Waves, Split Counter Coordinate}
\begin{displaydance}
\phBB{b1n,b3n,b2n,b4n} & \phBBBB{b1n,b3n,p..,p..,b2n,b4n,p..,p..} & \pvBBBB{p..,p..,p..,p..,b4w,b3w,b2w,b1w} & \phBB{b3w,b1w,b4w,b2w} \cr
\cbox{Before Call} & \cbox{After Adding Phantoms} & \cbox{After Split\\Counter Coordinate} & \cbox{Finished}\\
\end{displaydance}
\endexample

% \example{Mimic Beaus, Grand Swing Thru}
% \begin{displaydance}
% \phfour{b1n,b2n,b3n,b4n} & \pheight{b1n,p..,b2n,p..,b3n,p..,b4n,p..}\cr
% \cbox{Before Call} & \cbox{After Adding Phantoms}\cr
% \pheight{p..,p..,b1n,p..,b2n,p..,b3n,b4s} & \phfour{b1n,b2n,b3n,b4s}\cr
% \cbox{After Grand Swing Thru} & \cbox{Finished}\\
% \end{displaydance}
% \endexample

\example{Mimic Beaus of a Tidal Wave, Relay the Shadow}
\begin{displaydance}
\phfour{b1n,b2n,b3n,b4n} & \pheight{b1n,p..,b2n,p..,b3n,p..,b4n,p..} & \phBBBB{b2n,b4n,b1s,b3s,p..,p..,p..,p..} & \phBB{b2n,b4n,b1s,b3s} \cr
\cbox{Before Call} & \cbox{After Adding Phantoms} & \cbox{After Relay the Shadow} & \cbox{Finished}\\
\end{displaydance}
\endexample

Now let's consider some examples where
you have to be careful about the ending setup.

\example{Mimic Trailers of Lines, Load the Boat}
\begin{displaydance}
\phfour{b1n,b2s,b3s,b4n} & \phBBBB{p..,b1n,b2s,p..,b3s,p..,p..,b4n} &
\phBBBB{p..,b4e,b3w,p..,b2e,p..,p..,b1w} & \phfour{b4e,b3w,b2e,b1w} \cr
\cbox{Before Call} & \cbox{After Adding Phantoms} &
\cbox{After Load the Boat} & \cbox{Finished}\\
\end{displaydance}
\endexample

Note that the dancers collapsed the setup into a 1x4, rather than a 2x2, in
accordance with the principle that the resulting setup should be as close
as possible in shape and elongation axis to the original setup.
Contrast that with the example below:

\example{Mimic Beaus of Lines, Load the Boat}
\begin{displaydance}
\phBB{b1s,b3n,b2s,b4n} & \phBBBB{p..,b3n,b1s,p..,p..,b4n,b2s,p..} & \phBBBB{b2e,p..,p..,b4w,b1e,p..,p..,b3w} & \phBB{b2e,b4w,b1e,b3w} \cr
\cbox{Before Call} & \cbox{After Adding Phantoms} & \cbox{After Load the Boat} & \cbox{Finished}\\
\end{displaydance}
\endexample

Below is another example where you must be careful about the ending setup.

\example{Mimic Leads of Waves, Follow Your Leader}
\begin{displaydance}
\phfour{b1n,b2s,b3n,b4s} & \phBBBB{b1n,p..,p..,b2s,b3n,p..,p..,b4s} & \phBBBB{b4e,p..,b2e,p..,p..,b3w,p..,b1w} & \phfour{b4e,b2e,b3w,b1w} \cr
\cbox{Before Call} & \cbox{After Adding Phantoms} & \cbox{After Follow Your Leader} & \cbox{Finished}\\
\end{displaydance}
\endexample

Note that the dancers step sideways for the final adjustment,
in order to match the original 1x4.

% sue: I'm delaying discussion of transfer and checkmate to
% a later section, where we discuss ``leads'' vs ``lead 2''.

% Here is another example where you must be careful about the ending setup.
% Also, note that from columns, the designator ``leads'' refers to the
% leads of each box, not the first two in the column.
% 
% \example{Mimic Leads, Transfer the Column}
% \begin{displaydance}
% \phBB{b1n,b3n,b2s,b4s} & \pvBBBB{p..,b1n,b2s,p..,p..,b3n,b4s,p..} & \pvBBBB{p..,b4e,p..,b3w,b2e,p..,b1w,p..} & \phBB{b4e,b3w,b2e,b1w} \cr
% \cbox{Before Call} & \cbox{After Adding Phantoms} & \cbox{After Transfer the Column} & \cbox{Finished}\\
% \end{displaydance}
% \endexample
% 
% The setup could have been collapsed to a 1x4 wave
% but in this case we collapse to a 2x2 in order to preserve the shape
% of the starting setup.

Occasionally, you will find that there are multiple ways of putting
the setup back, and neither one exactly preserves the starting setup.
In this case, choose the setup that is as close as possible in
overall shape and direction of the long axis, if any.  For example,
suppose you start in a 1x4 and have a choice of ending in a 1x4
\emph{with the opposite elongation axis} or in a 2x2.
The 2x2 would be considered closer to the original
shape---half as wide and twice as deep,
rather than a quarter as wide and four times as deep.
Here is an example:

\example{Mimic Leads of Lines, Cast a Shadow}
\displayfour{\dancer{1}{n}\dancer{2}{n}\dancer{3}{s}\dancer{4}{s}}
{Before Call}
{\dancer{1}{n}\dancer{2}{n}\pdancer{}{x}\pdancer{}{x}\cr
 \pdancer{}{x}\pdancer{}{x}\dancer{3}{s}\dancer{4}{s}}
{After Adding Phantoms}
{\dancer{1}{w}\pdancer{}{x}\cr
 \dancer{2}{e}\pdancer{}{x}\cr
 \pdancer{}{x}\dancer{3}{w}\cr
 \pdancer{}{x}\dancer{4}{e}}
{After Cast a Shadow}
{\dancer{1}{w}\dancer{3}{w}\cr
 \dancer{2}{e}\dancer{4}{e}}
{After Entire Call}
\endexample

Sometimes it is simply not possible to squeeze out the phantoms.
In that case, that concept/call combination is not permissible.

\example{Mimic Leads of Lines, Out Roll Circulate (not permissible)}
\begin{displaydance}
\phfour{b1n,b2n,b3n,b4s} & \phBBBB{b1n,p..,b2n,p..,b3n,p..,p..,b4s} & \phBBBB{b2s,b4n,b3s,p..,p..,p..,b1s,p..} \cr
\cbox{Before Call} & \cbox{After Adding Phantoms} & \cbox{Out Roll Circulate,\\impossible to finish}\\
\end{displaydance}
\endexample

Since \emph{Mimic} versions of eight-person calls are four-person calls,
we can use these calls with various concepts that require four-person
calls.  Examples include Crazy, Checkpoint, Checkerboard, Triple Box, Z,
and many others.

\example{Z, Mimic Beaus, Regroup}
\begin{displaydance}
\pvBBB{...,b1n,b2n,b3s,b4s,...} & \phBB{b1n,b3s,b2n,b4s} & \phBBBB{b1n,p..,p..,b3s,b2n,p..,p..,b4s}\cr
\cbox{Before Call} & \cbox{Remove ``Z'' Offset} & \cbox{Add Phantoms}\cr
\pvBBBB{b4w,p..,p..,b2e,b3w,p..,p..,b1e} & \phBB{b2e,b1e,b4w,b3w} & \pvBBB{...,b2e,b4w,b1e,b3w,...}\cr
\cbox{Regroup} & \cbox{Remove Phantoms,\\back to 2x2} & \cbox{Restore ``Z''}\\
\end{displaydance}
\endexample

\example{1/2 Crazy, Mimic Leads of Columns, Wind the Bobbin}
\begin{displaydance}
\phBBBB{b1n,b5n,b2s,b6s,b3n,b7n,b4s,b8s} & \phBBBB{b6s,b2s,b4s,b8s,b1n,b5n,b7n,b3n}\cr
\cbox{Before Call} & \cbox{Finished}\\
\end{displaydance}
\endexample

Each part of the ``Crazy'' is performed separately,
with its own placement and removal of phantoms.

There are not many useful examples that start from diamonds; here is one.

\example{Mimic Leads, Lickety Split}
\begin{displaydance}
\pvdiamond{b1e,b2s,b3w,b4n} & \pvBBBB{b1e,p..,p..,b4n,b2s,p..,p..,b3w} & \pvBBBB{p..,b4e,b1w,p..,p..,b3e,b2w,p..} & \pvdiamond{b4e,b1w,b2w,b3e} \cr
\cbox{Before Call} & \cbox{After Adding Phantoms} & \cbox{After Lickety Split} & \cbox{Finished}\\
\end{displaydance}
\endexample

This example illustrates a subtle point:
when you add a phantom so that you become a lead,
you are a lead with respect
to the phantom you just added.  You are usually, but not always,
a lead with respect to the entire setup.  In this particular
example, not all dancers are leads of the 2x4 or leads of the box
on each side, but they are all
leads of the pair they form with their phantoms.

After doing the Lickety Split,
the dancers go back to diamond spots, since that makes the call a
non-shape-changer.
When danced in a single smooth flowing motion, this call is
Diamond Circulate and New Centers Roll.

So far, we have generally shown diagrams illustrating only the
minimum number of people required to do the \emph{Mimic} call.
In a square of eight people, there will generally be multiple
groups of either two or four people doing the \emph{Mimic} call,
just as multiple groups of people do typical two- or four-person calls
such as Single Wheel or Shakedown.
Our intention is that each of these groups does the \emph{Mimic} call
independently, just as they would do any other two- or four-person
calls independently.

Consider, for example, \emph{Mimic Leads, Shakedown}, a call
we introduced earlier.
This call is equivalent to
As Couples 1/4 Right and (individually) Roll Twice,
and it is straightforward as long as you only look at the
two-person starting setup.
However, imagine that another two-person starting setup is nearby.

\example{Mimic Leads, Shakedown}
\begin{displaydance}
\phBB{b1n,b3s,b2n,b4s} & \pvBBBB{b2n,b1n,p..,p..,p..,p..,b4s,b3s} & \pvBBBB{b1w,p..,b2w,p..,p..,b3e,p..,b4e} & \pvfour{b1w,b2w,b3e,b4e} \cr
\cbox{Before Call} & \cbox{After Adding Phantoms} & \cbox{After Shakedown} & \cbox{Finished}\\
\end{displaydance}
\endexample

Each couple does the entire \emph{Mimic} operation independently
of the other couple:
they add phantoms, they do the Shakedown, and they collapse 
the setup with \emph{the phantoms they added}.
They do \emph{not} collapse the setup
with the phantoms from the other couple also doing the Shakedown.
Just like Hinge, Shazam, Single Wheel, and other two person calls,
their behavior is exactly the same regardless of where the
other setups are.

Some dancers may be tempted to merge the two setups together,
making the above call finish in facing couples
(i.e. equivalent to a normal Shakedown).
This may be tempting because it would
make the \emph{overall, eight-person} setup closer in shape and
elongation
to the original \emph{overall, eight-person} setup.
However, we intend this rule
to apply \emph{independently and separately}
to each smaller setup doing the call, not to the overall, eight-person setup.
Within the group doing the call and consisting of the two
real people and the two phantoms they added, there is only one way to
remove the phantoms at the end, so there is no need to even apply
the ``closest possible shape and elongation'' rule.

If there is any ambiguity as to whom you are working with on a given
call, then the caller must use the full syntax.  For example, 
the caller could say \emph{Mimic Leads of a Box, Shakedown} to
emphasize that the dancers are to work in their box and only
consider collapsing the setup by using the phantoms in their box.

The issue is not limited to four-person base calls; below is an
example with an eight-person base call.

\example{Mimic Leads of Lines, Take 4}
\displayfour
{\dancer{1}{n}\dancer{2}{n}\dancer{3}{n}\dancer{4}{n}\cr
 \dancer{5}{s}\dancer{6}{s}\dancer{7}{s}\dancer{8}{s}}
{Before Call}
{\dancer{1}{n}\dancer{2}{n}\dancer{3}{n}\dancer{4}{n}\cr
 \pdancer{}{x}\pdancer{}{x}\pdancer{}{x}\pdancer{}{x}\cr
 \pdancer{}{x}\pdancer{}{x}\pdancer{}{x}\pdancer{}{x}\cr
 \dancer{5}{s}\dancer{6}{s}\dancer{7}{s}\dancer{8}{s}}
{After Adding Phantoms}
{\pdancer{}{x}\dancer{1}{w}\cr
 \pdancer{}{x}\dancer{2}{w}\cr
 \pdancer{}{x}\dancer{3}{w}\cr
 \pdancer{}{x}\dancer{4}{w}\cr
 \dancer{5}{e}\pdancer{}{x}\cr
 \dancer{6}{e}\pdancer{}{x}\cr
 \dancer{7}{e}\pdancer{}{x}\cr
 \dancer{8}{e}\pdancer{}{x}}
{After Take 4}
{\dancer{1}{w}\cr
 \dancer{2}{w}\cr
 \dancer{3}{w}\cr
 \dancer{4}{w}\cr
 \dancer{5}{e}\cr
 \dancer{6}{e}\cr
 \dancer{7}{e}\cr
 \dancer{8}{e}}
{Finished}
\endexample

Note that the correct ending formation is a 1x8, \emph{not} facing lines.
There is only one way to remove the phantoms, as long as you are
looking at your 2x4, containing four real people and the
four phantoms you added.

In summary, treat the calls as you would any other two- or four-person
call in square dancing.  
\emph{Mimic Leads, Shakedown} is a two-person call and
\emph{Mimic Leads, Take 4} is a four-person call.
The ``closest possible shape and elongation axis'' rule refers to the
smallest setup containing the people you are working with when doing the
call and the phantoms that you added; it does not refer to the
overall eight-person setup.
In the event of ambiguity, the caller
should name the setup explicitly.

\section{Specifying Setups}
\label{specifyingsetups}

In this section, we discuss in more detail when it is important
to explicitly name the setup.
We begin with two calls we introduced previously:
\emph{Mimic Leads of Lines, Link Up} and
\emph{Mimic Leads of Columns, Trade By}.
Consider the starting setup below, and assume that these
calls are given only with the shorter syntax, 
\emph{Mimic Leads}.

\displaytwo
{\dancer{1}{n}\dancer{2}{n}\dancer{3}{s}\dancer{4}{s}\cr
 \dancer{5}{n}\dancer{6}{n}\dancer{7}{s}\dancer{8}{s}}
{Before Call}
{\dancer{1}{n}\dancer{2}{n}\pdancer{}{x}\pdancer{}{x}\cr
 \pdancer{}{x}\pdancer{}{x}\dancer{3}{s}\dancer{4}{s}\cr
 \dancer{5}{n}\dancer{6}{n}\pdancer{}{x}\pdancer{}{x}\cr
 \pdancer{}{x}\pdancer{}{x}\dancer{7}{s}\dancer{8}{s}}
{\ctablebox{After Adding Phantoms\\for Mimic Leads}}

If the call is Link Up, it is done as though in Split Phantom Lines.
If it is Trade By, it is done as though in Split Phantom Columns.
Explicitly naming the setup is not technically required for these two calls,
since they have well-known and unambiguous starting formations.
However, in our experience dancers prefer to have the setup named,
since they are accustomed to having the setup named before most
eight-person calls.
We call ``Split Phantom Lines'' or ``Split Phantom Boxes'';
we don't call ``Split Phantom Setups'' (say, from a 2x4)
and expect them to figure out which one it must be.
Therefore, we recommend stating the formation explicitly
when calls such as the examples above are used from a 2x4 setup,
even if not technically required.

On some examples, use of the full syntax is required to
prevent ambiguity.
For example, suppose someone attempted to call
\emph{Mimic Leads, Circulate}, from parallel waves.
Everyone takes a small step forward, putting a phantom behind themselves.
But then, in which setup do they Circulate?  They could Circulate
in their Split Phantom Columns (working with people from their
original box), or they could Circulate in their Split Phantom Waves
(working with people from their original wave).  Some might even
argue they could do a Box Circulate, working with the
people in their original miniwaves.
This ambiguity is avoided by using the full syntax.

\example{Mimic Leads of Columns, Circulate}
\displaytwo
{\dancer{1}{n}\dancer{2}{s}\dancer{3}{n}\dancer{4}{s}\cr
 \dancer{5}{n}\dancer{6}{s}\dancer{7}{n}\dancer{8}{s}}
{Before Call}
{\dancer{1}{n}\pdancer{}{x}\dancer{3}{n}\pdancer{}{x}\cr
 \pdancer{}{x}\dancer{2}{s}\pdancer{}{x}\dancer{4}{s}\cr
 \dancer{5}{n}\pdancer{}{x}\dancer{7}{n}\pdancer{}{x}\cr
 \pdancer{}{x}\dancer{6}{s}\pdancer{}{x}\dancer{8}{s}}
{After Adding Phantoms}
\displaytwo
{\pdancer{}{x}\dancer{1}{s}\pdancer{}{x}\dancer{3}{s}\cr
 \dancer{5}{n}\pdancer{}{x}\dancer{7}{n}\pdancer{}{x}\cr
 \pdancer{}{x}\dancer{2}{s}\pdancer{}{x}\dancer{4}{s}\cr
 \dancer{6}{n}\pdancer{}{x}\dancer{8}{n}\pdancer{}{x}}
{After (column) Circulate}
{\dancer{5}{n}\dancer{1}{s}\dancer{7}{n}\dancer{3}{s}\cr
 \dancer{6}{n}\dancer{2}{s}\dancer{8}{n}\dancer{4}{s}}
{After Entire Call}
\endexample

\example{Mimic Leads of Waves, Circulate}
\displaytwo
{\dancer{1}{n}\dancer{2}{s}\dancer{3}{n}\dancer{4}{s}\cr
 \dancer{5}{n}\dancer{6}{s}\dancer{7}{n}\dancer{8}{s}}
{Before Call}
{\dancer{1}{n}\pdancer{}{x}\dancer{3}{n}\pdancer{}{x}\cr
 \pdancer{}{x}\dancer{2}{s}\pdancer{}{x}\dancer{4}{s}\cr
 \dancer{5}{n}\pdancer{}{x}\dancer{7}{n}\pdancer{}{x}\cr
 \pdancer{}{x}\dancer{6}{s}\pdancer{}{x}\dancer{8}{s}}
{After Adding Phantoms}
\displaytwo
{\pdancer{}{x}\dancer{3}{s}\pdancer{}{x}\dancer{1}{s}\cr
 \dancer{4}{n}\pdancer{}{x}\dancer{2}{n}\pdancer{}{x}\cr
 \pdancer{}{x}\dancer{7}{s}\pdancer{}{x}\dancer{5}{s}\cr
 \dancer{8}{n}\pdancer{}{x}\dancer{6}{n}\pdancer{}{x}}
{After (wave-type) Circulate}
{\dancer{4}{n}\dancer{3}{s}\dancer{2}{n}\dancer{1}{s}\cr
 \dancer{8}{n}\dancer{7}{s}\dancer{6}{n}\dancer{5}{s}}
{After Entire Call}
\endexample

\example{Mimic Leads of a Box, Circulate}
\displaytwo
{\dancer{1}{n}\dancer{2}{s}\dancer{3}{n}\dancer{4}{s}\cr
 \dancer{5}{n}\dancer{6}{s}\dancer{7}{n}\dancer{8}{s}}
{Before Call}
{\dancer{1}{n}\pdancer{}{x}\dancer{3}{n}\pdancer{}{x}\cr
 \pdancer{}{x}\dancer{2}{s}\pdancer{}{x}\dancer{4}{s}\cr
 \dancer{5}{n}\pdancer{}{x}\dancer{7}{n}\pdancer{}{x}\cr
 \pdancer{}{x}\dancer{6}{s}\pdancer{}{x}\dancer{8}{s}}
{After Adding Phantoms}
\displaytwo
{\pdancer{}{x}\dancer{1}{s}\pdancer{}{x}\dancer{3}{s}\cr
 \dancer{2}{n}\pdancer{}{x}\dancer{4}{n}\pdancer{}{x}\cr
 \pdancer{}{x}\dancer{5}{s}\pdancer{}{x}\dancer{7}{s}\cr
 \dancer{6}{n}\pdancer{}{x}\dancer{8}{n}\pdancer{}{x}}
{After (Box) Circulate}
{\dancer{2}{n}\dancer{1}{s}\dancer{4}{n}\dancer{3}{s}\cr
 \dancer{6}{n}\dancer{5}{s}\dancer{8}{n}\dancer{7}{s}}
{After Entire Call}
\endexample

Note also that there is no default setup for situations where
the call can be done in either lines or columns.
If the call can be done in two different setups and produces different
results, and the caller does not specify the setup,
then the call is ambiguous.
\footnote{A previous version of this paper attempted to establish
the box as the default starting setup (by analogy with Central).
This would imply that columns would be preferred over lines
for Mimic Leads, and lines would be preferred over columns for
Mimic Beaus or Belles.  We now think it is better to have
no default and force callers to explicitly name the setup.}

The call Circulate provides a challenge in identifying the setup,
but is straightforward when collapsing the setup at the end.
The call Counter Rotate is a little harder.  See if you can
figure out the example calls below.

\example{Mimic Leads of Columns, Counter Rotate}
\displaytwo
{\dancer{1}{n}\dancer{2}{s}\dancer{3}{n}\dancer{4}{s}\cr
 \dancer{5}{n}\dancer{6}{s}\dancer{7}{n}\dancer{8}{s}}
{Before Call}
{\dancer{1}{n}\pdancer{}{x}\dancer{3}{n}\pdancer{}{x}\cr
 \pdancer{}{x}\dancer{2}{s}\pdancer{}{x}\dancer{4}{s}\cr
 \dancer{5}{n}\pdancer{}{x}\dancer{7}{n}\pdancer{}{x}\cr
 \pdancer{}{x}\dancer{6}{s}\pdancer{}{x}\dancer{8}{s}}
{After Adding Phantoms}
\displaytwo
{\pdancer{}{x}\dancer{5}{e}\pdancer{}{x}\dancer{1}{e}
     \pdancer{}{x}\dancer{7}{e}\pdancer{}{x}\dancer{3}{e}\cr
 \dancer{6}{w}\pdancer{}{x}\dancer{2}{w}\pdancer{}{x}
     \dancer{8}{w}\pdancer{}{x}\dancer{4}{w}\pdancer{}{x}}
{After (column) Counter Rotate}
{\dancer{5}{e}\dancer{1}{e}\dancer{7}{e}\dancer{3}{e}\cr
 \dancer{6}{w}\dancer{2}{w}\dancer{8}{w}\dancer{4}{w}}
{After Entire Call}
\endexample

\example{Mimic Leads of Waves, Counter Rotate}
\displaytwo
{\dancer{1}{n}\dancer{2}{s}\dancer{3}{n}\dancer{4}{s}\cr
 \dancer{5}{n}\dancer{6}{s}\dancer{7}{n}\dancer{8}{s}}
{Before Call}
{\dancer{1}{n}\pdancer{}{x}\dancer{3}{n}\pdancer{}{x}\cr
 \pdancer{}{x}\dancer{2}{s}\pdancer{}{x}\dancer{4}{s}\cr
 \dancer{5}{n}\pdancer{}{x}\dancer{7}{n}\pdancer{}{x}\cr
 \pdancer{}{x}\dancer{6}{s}\pdancer{}{x}\dancer{8}{s}}
{After Adding Phantoms}
\displaytwo
{\pdancer{}{x}\dancer{1}{e}\cr
 \dancer{3}{w}\pdancer{}{x}\cr
 \pdancer{}{x}\dancer{2}{e}\cr
 \dancer{4}{w}\pdancer{}{x}\cr
 \pdancer{}{x}\dancer{5}{e}\cr
 \dancer{7}{w}\pdancer{}{x}\cr
 \pdancer{}{x}\dancer{6}{e}\cr
 \dancer{8}{w}\pdancer{}{x}}
{After (wave-type) Counter Rotate}
{\dancer{3}{w}\dancer{1}{e}\cr
 \dancer{4}{w}\dancer{2}{e}\cr
 \dancer{7}{w}\dancer{5}{e}\cr
 \dancer{8}{w}\dancer{6}{e}}
{After Entire Call}
\endexample

In the last example above, we used the ``closest possible elongation'' rule
(\emph{within each group doing the call}) to choose couples back-to-back,
rather than a wave, as the result of the call.
This makes lines back-to-back be the overall result,
rather than a tidal wave.

\example{Mimic Leads of a Box, Counter Rotate}
\displaytwo
{\dancer{1}{n}\dancer{2}{s}\dancer{3}{n}\dancer{4}{s}\cr
 \dancer{5}{n}\dancer{6}{s}\dancer{7}{n}\dancer{8}{s}}
{Before Call}
{\dancer{1}{n}\pdancer{}{x}\dancer{3}{n}\pdancer{}{x}\cr
 \pdancer{}{x}\dancer{2}{s}\pdancer{}{x}\dancer{4}{s}\cr
 \dancer{5}{n}\pdancer{}{x}\dancer{7}{n}\pdancer{}{x}\cr
 \pdancer{}{x}\dancer{6}{s}\pdancer{}{x}\dancer{8}{s}}
{After Adding Phantoms}
\displaytwo
{\pdancer{}{x}\dancer{1}{e}\pdancer{}{x}\dancer{3}{e}\cr
 \dancer{2}{w}\pdancer{}{x}\dancer{4}{w}\pdancer{}{x}\cr
 \pdancer{}{x}\dancer{5}{e}\pdancer{}{x}\dancer{7}{e}\cr
 \dancer{6}{w}\pdancer{}{x}\dancer{8}{w}\pdancer{}{x}}
{After (Box) Counter Rotate}
{\dancer{2}{w}\dancer{1}{e}\dancer{4}{w}\dancer{3}{e}\cr
 \dancer{6}{w}\dancer{5}{e}\dancer{8}{w}\dancer{7}{e}}
{After Entire Call}
\endexample

In the examples we have shown so far, the setup was a 4x4
after adding phantoms.  It is also possible for the setup
to be a 2x8 at that point.
In most of these cases, the dancers will be doing the call
in each 2x4 (as in Split Phantom Boxes),
but it is also possible to work in each 1x8
(as in Twin Phantom Tidal Lines/Columns.)

\example{Mimic Belles of a Tidal Wave, Relay the Shadow}
\displaytwo
{\dancer{1}{n}\dancer{2}{n}\dancer{3}{n}\dancer{4}{n}\cr
 \dancer{5}{s}\dancer{6}{s}\dancer{7}{s}\dancer{8}{s}}
{Before Call}
{\pdancer{}{x}\dancer{1}{n}\pdancer{}{x}\dancer{2}{n}\pdancer{}{x}\dancer{3}{n}\pdancer{}{x}\dancer{4}{n}\cr
 \dancer{5}{s}\pdancer{}{x}\dancer{6}{s}\pdancer{}{x}\dancer{7}{s}\pdancer{}{x}\dancer{8}{s}\pdancer{}{x}}
{After Adding Phantoms}
\displaytwo
{\pdancer{}{x}\pdancer{}{x}\dancer{4}{s}\dancer{3}{n}\cr
 \pdancer{}{x}\pdancer{}{x}\dancer{2}{s}\dancer{1}{n}\cr
 \dancer{8}{s}\dancer{7}{n}\pdancer{}{x}\pdancer{}{x}\cr
 \dancer{6}{s}\dancer{5}{n}\pdancer{}{x}\pdancer{}{x}}
{After Relay the Shadow}
{\dancer{4}{s}\dancer{3}{n}\cr
 \dancer{2}{s}\dancer{1}{n}\cr
 \dancer{8}{s}\dancer{7}{n}\cr
 \dancer{6}{s}\dancer{5}{n}}
{After Entire Call}
\endexample

Remember that the removal of phantoms is done in the result of the call
from each tidal wave.

\example{Mimic Beaus of Waves, Follow Your Leader}
\displaytwo
{\dancer{1}{n}\dancer{2}{n}\dancer{3}{s}\dancer{4}{s}\cr
 \dancer{5}{n}\dancer{6}{n}\dancer{7}{s}\dancer{8}{s}}
{Before Call}
{\dancer{1}{n}\pdancer{}{x}\dancer{2}{n}\pdancer{}{x}\pdancer{}{x}\dancer{3}{s}\pdancer{}{x}\dancer{4}{s}\cr
 \dancer{5}{n}\pdancer{}{x}\dancer{6}{n}\pdancer{}{x}\pdancer{}{x}\dancer{7}{s}\pdancer{}{x}\dancer{8}{s}}
{After Adding Phantoms}
\displaytwo
{\pdancer{}{x}\pdancer{}{x}\pdancer{}{x}\pdancer{}{x}\dancer{8}{e}\dancer{7}{e}\dancer{3}{e}\dancer{4}{e}\cr
 \dancer{5}{w}\dancer{6}{w}\dancer{2}{w}\dancer{1}{w}\pdancer{}{x}\pdancer{}{x}\pdancer{}{x}\pdancer{}{x}}
{After Follow Your Leader}
{\dancer{5}{w}\dancer{6}{w}\dancer{2}{w}\dancer{1}{w}\dancer{8}{e}\dancer{7}{e}\dancer{3}{e}\dancer{4}{e}}
{After Entire Call}
\endexample

\section{Choice of Designators}

The Mimic concept is most intuitive if the role designated matches
the way people think about the call.  For example, most dancers perceive
Follow Your Leader as having leads' and trailers' parts.
So, \emph{Mimic Leads} or \emph{Mimic Trailers} would be the
most intuitive uses of \emph{Mimic} with this call
(at least when starting the call).
However, the definition does not require that the usage
match the verbal definition of the call.

\example{Mimic Belles of Waves, Follow Your Leader}
\displayfour
{\dancer{1}{n}\dancer{2}{s}\cr
 \dancer{3}{n}\dancer{4}{s}}
{Before Call}
{\pdancer{}{x}\dancer{1}{n}\dancer{2}{s}\pdancer{}{x}\cr
 \pdancer{}{x}\dancer{3}{n}\dancer{4}{s}\pdancer{}{x}}
{After Adding Phantoms}
{\pdancer{}{x}\dancer{2}{w}\dancer{4}{w}\pdancer{}{x}\cr
 \pdancer{}{x}\dancer{1}{e}\dancer{3}{e}\pdancer{}{x}}
 {After Follow Your Leader}
{\dancer{2}{w}\dancer{4}{w}\cr
 \dancer{1}{e}\dancer{3}{e}}
{After Entire Call}
\endexample

The dancers add phantoms in such a way that the real people become
belles.  In this case, phantoms are added to the outside.  The phantoms'
facing directions are not specified by the Mimic concept, but here you
could assume the phantoms make waves (if you care),
since that would be required for Follow Your Leader.
This example could also be called
\emph{Mimic Centers, Follow Your Leader} since all the phantoms were
added to the outside.
However, with \emph{Mimic Centers}, dancers would have to take an extra
thinking step to decide whether the phantoms should be added to
make lines or columns.
\emph{Mimic Centers of Waves} might be best, since it makes
the starting setup clear immediately.
Any of these examples would be considered acceptable.

Another example that is reasonable to call multiple ways is Strut Right.

\example{Mimic Leads of Columns, Strut Right}
\displayfour{\dancer{1}{s}\dancer{2}{s}\cr
 \dancer{3}{n}\dancer{4}{n}}
{Before Call}
{\pdancer{}{x}\pdancer{}{x}\cr
 \dancer{1}{s}\dancer{2}{s}\cr
 \dancer{3}{n}\dancer{4}{n}\cr
 \pdancer{}{x}\pdancer{}{x}}
{\ctablebox{After Adding\\Phantoms}}
{\dancer{1}{n}\pdancer{}{x}\dancer{2}{n}\pdancer{}{x}
 \pdancer{}{x}\dancer{3}{s}\pdancer{}{x}\dancer{4}{s}}
 {After Strut Right}
{\dancer{1}{n}\dancer{2}{n}\dancer{3}{s}\dancer{4}{s}}
{After Entire Call}
\endexample

This call is the same as Pass Thru and Turn to a Line.
\emph{Mimic Leads of Columns, Strut Right} could also be called
\emph{Mimic Centers of Columns, Strut Right}.
\emph{Mimic Trailers of Columns, Strut Right}, which is the same as Veer Right,
could also be called \emph{Mimic Ends of Columns, Strut Right}.
Some dancers will prefer \emph{Mimic Leads} because they know to immediately
add a phantom behind themselves.
Some dancers will prefer \emph{Mimic Centers} because they think of the call
as having centers' and ends' parts.

The use of \emph{Leads} or \emph{Trailers} is somewhat unnatural for
calls like Transfer or Checkmate.  A more natural thing would be to
mimic the positions that do similar parts of the call, that is, the
first two in the column or the last two.  To get this more intuitive
use of Mimic, we need some new designators: \emph{Mimic First Two}
and \emph{Mimic Last Two} for columns.
We also have \emph{Mimic Leftmost Two}
and \emph{Mimic Rightmost Two} for lines.

\example{Mimic First Two, Checkmate}
\displayfour
{\dancer{1}{n}\dancer{2}{s}\cr
 \dancer{3}{n}\dancer{4}{s}}
{Before Call}
{\dancer{1}{n}\pdancer{}{x}\cr
 \dancer{3}{n}\pdancer{}{x}\cr
 \pdancer{}{x}\dancer{2}{s}\cr
 \pdancer{}{x}\dancer{4}{s}}
{After Adding Phantoms}
{\dancer{4}{e}\pdancer{}{x}\cr
 \dancer{2}{e}\pdancer{}{x}\cr
 \pdancer{}{x}\dancer{3}{w}\cr
 \pdancer{}{x}\dancer{1}{w}}
 {After Checkmate}
{\dancer{4}{e}\dancer{3}{w}\cr
 \dancer{2}{e}\dancer{1}{w}}
{After Entire Call}
\endexample

The dancers move forward to become \#1 and \#2 in the column,
rather than becoming leads in each box.
The rest of the call proceeds as before.

If the call had been \emph{Mimic Leads}, then you should do the
following (less intuitive) usage.

\example{Mimic Leads of Columns, Checkmate}
\displayfour{\dancer{1}{n}\dancer{2}{s}\cr
 \dancer{3}{n}\dancer{4}{s}}{Before Call}
{\dancer{1}{n}\pdancer{}{x}\cr
 \pdancer{}{x}\dancer{2}{s}\cr
 \dancer{3}{n}\pdancer{}{x}\cr
 \pdancer{}{x}\dancer{4}{s}}
{After Adding Phantoms}
{\dancer{4}{e}\dancer{3}{e}\cr
 \pdancer{}{x}\pdancer{}{x}\cr
 \pdancer{}{x}\pdancer{}{x}\cr
 \dancer{2}{w}\dancer{1}{w}\cr}
 {After Checkmate}
{\dancer{4}{e}\dancer{3}{e}\cr
 \dancer{2}{w}\dancer{1}{w}\cr}
{After Entire Call}
\endexample

Below is an example using Leftmost 2:

\example{Mimic Leftmost 2, Link Up}
\displayfour
{\dancer{1}{n}\dancer{2}{n}\cr
 \dancer{3}{s}\dancer{4}{s}}
{Before Call}
{\dancer{1}{n}\dancer{2}{n}\pdancer{}{x}\pdancer{}{x}\cr
 \pdancer{}{x}\pdancer{}{x}\dancer{3}{s}\dancer{4}{s}}
{After Adding Phantoms}
{\dancer{2}{s}\pdancer{}{x}\pdancer{}{x}\dancer{4}{n}\cr
 \dancer{1}{s}\pdancer{}{x}\pdancer{}{x}\dancer{3}{n}}
{After Link Up}
{\dancer{2}{s}\dancer{4}{n}\cr
 \dancer{1}{s}\dancer{3}{n}\cr}
{After Entire Call}
\endexample

The dancers first slide to the left as a couple.
Then they do the link up and collapse the setup,
as before.  Conversely, if the call were \emph{Mimic Beaus},
each dancer would individually put one phantom on his or her right hand.
The setup would then look like right hand waves (or lines facing out),
and (presumably) some different call would be used.

\example{Mimic Leftmost 2, Load the Boat}
\begin{displaydance}
\phBB{b1s,b3n,b2s,b4n} & \phBBBB{p..,b3n,p..,b4n,b1s,p..,b2s,p..} & \phBBBB{b2e,p..,b1w,p..,p..,b4e,p..,b3w} & \phBB{b2e,b4e,b1w,b3w} \cr
\cbox{Before Call} & \cbox{After Adding Phantoms} & \cbox{After Load the Boat} & \cbox{Finished}\\
\end{displaydance}
\endexample

There are cases in which the designators ``first/last/leftmost/rightmost 4''
could be used.

\example{Mimic Leftmost 4 (of a 1x8), Grand Mix}
\displaytwo
{\dancer{1}{s}\dancer{2}{s}\dancer{3}{s}\dancer{4}{s}\cr
 \dancer{5}{n}\dancer{6}{n}\dancer{7}{n}\dancer{8}{n}}
{Before Call}
{\pdancer{}{x}\pdancer{}{x}\pdancer{}{x}\pdancer{}{x}\dancer{1}{s}\dancer{2}{s}\dancer{3}{s}\dancer{4}{s}\cr
 \dancer{5}{n}\dancer{6}{n}\dancer{7}{n}\dancer{8}{n}\pdancer{}{x}\pdancer{}{x}\pdancer{}{x}\pdancer{}{x}}
{After Adding Phantoms}
\displaytwo
{\pdancer{}{x}\dancer{1}{s}\pdancer{}{x}\dancer{3}{s}\pdancer{}{x}\dancer{4}{n}\pdancer{}{x}\dancer{2}{n}\cr
 \dancer{7}{s}\pdancer{}{x}\dancer{5}{s}\pdancer{}{x}\dancer{6}{n}\pdancer{}{x}\dancer{8}{n}\pdancer{}{x}}
{After Grand Mix}
{\dancer{1}{s}\dancer{3}{s}\dancer{4}{n}\dancer{2}{n}\cr
 \dancer{7}{s}\dancer{5}{s}\dancer{6}{n}\dancer{8}{n}}
{After Entire Call}
\endexample

It is also possible to create eight-person \emph{Mimic} calls,
but you need to use a base call that would require 16 people.
The following example could be called either
\emph{Mimic First 4 of 2x8 Columns, 4x4 Transfer the Column},
or \emph{4x4, Mimic First 2 of Columns, Transfer the Column}.
This call is equivalent to 4x0 Transfer the Column.

\example{Mimic First 4 of 2x8 Columns, 4x4 Transfer the Column}
\displaytwo
{\dancer{1}{e}\dancer{2}{e}\dancer{3}{e}\dancer{4}{e}\cr
 \dancer{5}{w}\dancer{6}{w}\dancer{7}{w}\dancer{8}{w}}
{Before Call}
{\pdancer{}{x}\pdancer{}{x}\pdancer{}{x}\pdancer{}{x}\dancer{1}{e}\dancer{2}{e}\dancer{3}{e}\dancer{4}{e}\cr
 \dancer{5}{w}\dancer{6}{w}\dancer{7}{w}\dancer{8}{w}\pdancer{}{x}\pdancer{}{x}\pdancer{}{x}\pdancer{}{x}}
{After Adding Phantoms}
\displaytwo
{\pdancer{}{x}\dancer{8}{s}\pdancer{}{x}\dancer{7}{s}\pdancer{}{x}\dancer{6}{s}\pdancer{}{x}\dancer{5}{s}\cr
 \dancer{4}{n}\pdancer{}{x}\dancer{3}{n}\pdancer{}{x}\dancer{2}{n}\pdancer{}{x}\dancer{1}{n}\pdancer{}{x}}
{After 4x4 Transfer the Column}
{\dancer{8}{s}\dancer{7}{s}\dancer{6}{s}\dancer{5}{s}\cr
 \dancer{4}{n}\dancer{3}{n}\dancer{2}{n}\dancer{1}{n}}
{After Entire Call}
\endexample

One could also use the designators ``first/last/leftmost/rightmost 3''
in suitable formations.

There is one more issue on the choice of designators that is worth
pointing out and applies specifically to using \emph{Mimic Leads}
on line-of-4 (or line-of-2) calls.
Although we have said that \emph{Mimic} calls require half
the number of people as the base call, that is only true
if the phantoms are added in a way that we will call ``useful''.
For example, if the call is \emph{Mimic Beaus} applied to a line-of-4 call
such as Tag the Line or Flip Back,
the phantoms are added in a way that is useful:
a two-person setup (i.e., a couple) cannot do
a line-of-4 call by itself, but it can do the call after phantoms are
added between them so as to make lines.
However, if the call is \emph{Mimic Leads} applied to a line-of-4 call,
the added phantoms are not useful in that sense:
they do not turn any two-person setups
into lines of 4.  Instead, they just create more lines of 4.

\example{Mimic Leads of Waves, Flip Back}
\begin{displaydance}
\phfour{b1n,b6s,b2n,b5s} & \phBBBB{b1n,p..,p..,b6s,b2n,p..,p..,b5s} & \pvBBBB{b2e,p..,b1w,p..,p..,b5e,p..,b6w} & \phBB{b5e,b6w,b2e,b1w} \cr
\cbox{Before Call} & \cbox{After Adding Phantoms} & \cbox{After Flip Back} & \cbox{Finished}\\
\end{displaydance}
\endexample

Another way to look at this is to realize that this example essentially treats
Flip Back as an eight-person call done from parallel waves.
A single wave does not have leads and trailers, but parallel
waves do.  You adjust from a single line of 4 to create the
parallel wave setup.

\emph{Mimic} calls using ``non-useful'' phantoms
tend to be less intuitive since
the rule of using half the number of people as the base call does not
apply.  They also tend to be not very useful choreographically;
the net result is usually the same as doing the original call
without the \emph{Mimic} concept.  We discourage their use,
much as we would discourage ``Triple Boxes Working Together, Scoot Back'',
but we do not explicitly label them improper.  Beware, though,
that when thinking about which real people you are working with
and which phantoms you can merge with, you must treat Flip Back as if
it were an eight-person call.  The phantoms you may merge with include
the phantoms in your wave (when starting to do the Flip Back) \emph{and}
the ones you added (even though you are not doing the Flip Back with
them).  \emph{Mimic Leads, Flip Back} is a four-person call even though
\emph{Mimic Beaus, Flip Back} is a two-person call.

% examples from an earlier draft of the paper, no longer used
% 
% \section{More examples}
% 
% \example{Mimic Beaus, Track 2}
% \displayfour
% {\dancer{1}{w}\dancer{2}{w}\dancer{3}{e}\dancer{4}{e}}
% {Before Call}
% {\pdancer{}{x}\pdancer{}{x}\dancer{3}{e}\dancer{4}{e}\cr
%  \dancer{1}{w}\dancer{2}{w}\pdancer{}{x}\pdancer{}{x}}
% {After Adding Phantoms}
% {\dancer{2}{e}\dancer{1}{e}\cr
%  \pdancer{}{x}\pdancer{}{x}\cr
%  \pdancer{}{x}\pdancer{}{x}\cr
%  \dancer{4}{w}\dancer{3}{w}}
% {After Track 2}
% {\dancer{2}{e}\dancer{1}{e}\cr
%  \dancer{4}{w}\dancer{3}{w}}
% {After Entire Call}
% \endexample
% 
% \example{Mimic Beaus, Single Wheel}
% \displayfour{\dancer{1}{n}}{Before Call}
%             {\dancer{1}{n}\pdancer{}{x}}{After Adding Phantoms}
% 	    {\dancer{1}{s}\cr\pdancer{}{x}}{After Single Wheel}
% 	    {\dancer{1}{s}}{After Entire Call}
% \endexample
% 
% This is a U-Turn Back to the right.
% Now consider \emph{Mimic Belles, Single Wheel}.
% You might initially think this call is ambiguous, because Mimic does not
% specify the facing directions of phantoms.  So, the real person does not
% know whether to go in front of the phantom or behind.  However, the call
% is not ambiguous, since that phantom will be eliminated anyway.
% \emph{Mimic Belles, Single Wheel} is just a U-Turn back to the left.

\section{Conclusions}

The original motivation for this concept was to provide a way of
doing the leads' part of the call even if you were not a lead,
or doing the beaus' part of the call even if you were not a beau.
The Mimic concept provides many examples of this type, allowing, for
example, everyone to do only the leads' part or only the trailers'
part of calls such as Link Up, Keep Busy, or Wind the Bobbin.
Mimic also allows everyone to do only the beaus' part or
only the belles' part of calls such as Track 2.

The definition of the concept is much broader than the above
mission would suggest, however.
It is often possible to use \emph{Mimic Leads}
with calls that you normally think
of as having centers' and ends' parts, or \emph{Mimic Beaus} with
calls you normally think of as having leads' and trailers' parts.
You can also use Mimic with calls such as Cast a Shadow,
which might be considered to have a lead centers' part, a trailing
centers' part, and two other parts.
Mimic simply defines an adjustment; you do the call normally
after making the adjustment.  Nothing in the definition depends on
the verbal definition of the call (except for defining the setups
from which the call can be done).

Mimic also provides a way to better define calls previously
called as ``Nx0'' or ``0xN'' (e.g. 4x0 Transfer the Column).
While Nx0/0xN works well on some column calls such as
Transfer and Checkmate, 
it tends to be ambiguous when applied more generally.
Mimic works better by explicitly naming the role each person has
and the setup in which they are working.
We recommend that Mimic be used instead of Nx0/0xN in most examples
other than traditional MxN column calls.

The impact of Mimic is to create many more two- and four-person calls
than we currently have.  Mimic calls generally require half the number of
people as the original call.  So, by using Mimic, we can create one
or more four-person calls for many of the eight-person calls we already
have, and one or more two-person calls for many of the four-person
calls we already have.  The only other similar concepts we can think of
are Single and Central.  However, these concepts are much more restrictive
and only apply to certain types of calls.  Mimic is more general and
presumably generates many more new calls.





\section{Acknowledgments}

We would like to thank Andy Latto, Eric Brosius, and Will and Mary Leland
for helpful suggestions and comments on an earlier draft of this paper.

\newpage
\section*{Practice Sequences}
\bigskip\bigskip

% no longer using multicols, some call names are too long, would split
% across multiple lines
% this is the syntax if you want multicols back in
% \begin{multicols}{2}
% \end{multicols}

\sequence{Two-person calls}
HEADS: pass the ocean \\
chain reaction \\
spin the top and spread \\
switch the wave \\
MIMIC LEADS, wheel the ocean \\
explode the top \\
MIMIC BEAUS, follow your neighbor \\
invert the column 1/2 \\
detour \\
recoil \\
MIMIC BEAUS, shazam \\
spin the top \\
in roll circulate \\
3/4 cast and relay \\
right and left grand  (7/8 promenade)
\endsequence

\bigskip

\sequence{More two-person calls}
SIDES: square thru 2\\
pass the axle\\
MIMIC BEAUS, cross and turn\\
step and slide\\
MIMIC BEAUS, peel off\\
plan ahead\\
strut right\\
flip the line\\
MIMIC BELLES, reach out\\
rotary spin\\
switch the wave\\
MIMIC LEADS, right roll to a wave\\
link up\\
cross your neighbor\\
promenade  (1/8 promenade)
\endsequence

\bigskip
\sequence{Two-person wave calls}
HEADS: split dixie style \\
acey deucey \\
recycle \\
MIMIC BELLES, left swing thru \\
circulate \\
exchange the boxes \\
walk and dodge \\
MIMIC BEAUS, flip the line \\
here comes the judge \\
right and left thru \\
dixie sashay \\
1/4 wheel the ocean \\
trade the wave \\
left allemande  (1/4 promenade)
\endsequence

\bigskip


\sequence{Leads of Lines}
HEADS: pass the ocean\\
extend\\
TANDEM TWOSOME, spin the top\\
MIMIC LEADS OF LINES, link up\\
counter rotate \\
follow thru \\
trade circulate \\
MIMIC LEADS OF LINES, mini busy \\
chain the square \\
flare out to a line \\
MIMIC LEADS OF LINES, keep busy \\
trade circulate \\
right and left grand  (3/8 promenade)
\endsequence

\sequence{Leads/Trailers of Waves}
SIDES: turn thru\\
wave the boys\\
tag the top\\
MIMIC LEADS OF WAVES, finish perk up\\
wind the bobbin\\
left swing thru \\
criss cross the deucey \\
MIMIC LEADS OF WAVES, criss cross the deucey\\
explode the wave \\
chase right \\
MIMIC TRAILERS OF WAVES, follow your neighbor \\
slip and slide \\
disband \\
left allemande  (1/2 promenade)
\endsequence

\sequence{Beaus/Belles}
HEADS: swap the top \\
extend \\
reflected flip the line \\
MIMIC BEAUS OF COLUMNS, track 2 \\
walk out to a wave \\
acey deucey \\
switch the wave and roll \\
MIMIC BEAUS OF COLUMNS, fancy \\
with the flow \\
split counter rotate \\
MIMIC BELLES OF COLUMNS, counter rotate \\
TANDEM TWOSOME, single polly wally \\
TANDEM, cross back \\
MIMIC BELLES, cross back \\
right and left grand  (7/8 promenade)
\endsequence

\sequence{Leads/Trailers of Columns}
HEADS: wheel thru\\
MIMIC LEADS OF COLUMNS, strut right\\
scatter circulate\\
recycle\\
MIMIC TRAILERS OF COLUMNS, strut left\\
link up\\
left 1/4 mix\\
MIMIC LEADS OF COLUMNS, wind the bobbin\\
trade the deucey\\
1/4 thru\\
MIMIC TRAILERS OF COLUMNS, wind the bobbin\\
out roll circulate; acey deucey\\
right and left grand  (7/8 promenade)\\
\endsequence

\sequence{Collapsing Rules}
HEADS: pass the ocean\\
extend \\
lock it \\
MIMIC LEADS OF WAVES, follow your leader (note 1x8 ending)\\
single file recycle\\
the gamut\\
AS COUPLES, 1/4 thru\\
MIMIC LEADS OF LINES, acey deucey\\
COUPLES TWOSOME, slip\\
cross roll\\
MIMIC LEADS OF WAVES, swing o late\\
1/4 wheel the ocean\\
bingo\\
right and left grand  (3/8 promenade)\\
\endsequence

\sequence{More 1x4 endings}
HEADS: wheel thru\\
pass the ocean\\
MIMIC LEADS OF LINES, chuck a luck (note 1x8 ending)\\
% 3G> 4B< 3B> 4G< 2G> 1B< 2B> 1G<
pass and roll\\
circulate\\
grand 1/4 thru\\
MIMIC LEADS OF WAVES, trade circulate\\
% 3BV 1G^ 4GV 2B^ 4BV 2G^ 3GV 1B^
grand left swing thru\\
MIMIC TRAILERS OF WAVES, scoot chain thru\\
% 2GV 2B^ 3BV 1G^ 3GV 1B^ 4BV 4G^
along\\
trade circulate\\
reflected flip the line\\
MIMIC BEAUS OF COLUMNS, invert the column 1/2\\
open up the column\\
left allemande  (1/4 promenade)\\
\endsequence

\sequence{Harder collapsing}
SIDES: split grand chain 8 with the flow\\
cross lock it\\
switch the wave\\
MIMIC LEADS, gee whiz (note box ending) \\
counter rotate\\
explode the wave\\
MIMIC LEADS, wheel the ocean (note two-person call)\\
\displayone{
\dancer{6}{s}\dancer{5}{s}\cr
\dancer{7}{s}\dancer{4}{s}\cr
\dancer{8}{n}\dancer{3}{n}\cr
\dancer{1}{n}\dancer{2}{n}}
{}
double pass thru\\
reset 1/2 \\
CENTERS: pass thru\\
team up\\
circulate\\
split counter rotate\\
MIMIC LEADS OF COLUMNS, finish wave the beaus \\
\displayone{
 \dancer{8}{n}\dancer{3}{w}\dancer{5}{n}\dancer{2}{w}\cr
 \dancer{6}{e}\dancer{1}{s}\dancer{7}{e}\dancer{4}{s}
}{}
follow to a diamond\\
relay the shadow\\
spin the top ers choice\\
right and left grand (1/8 promenade)
\endsequence

\sequence{More Designators}
HEADS: split swap \\
touch 1/4 \\
counter rotate \\
MIMIC FIRST 2 OF COLUMNS, checkmate \\
CENTERS: right roll to a wave \\
scoot and little more \\
MIMIC LAST 2 OF COLUMNS, transfer the column \\
the axle \\
chisel thru \\
plan ahead \\
CENTER 4: MIMIC LEFTMOST 2, link up \\
THE PULLEY BUT, MIMIC FIRST 2 OF COLUMNS, 2 steps at a time \\
relay the shadow \\
relay the top, star 1/2 \\
right and left grand  (1/8 promenade)
\endsequence

\end{document}

%-----------------
%
%\emph{Mimic Beaus, Flip Back} is a two-person call, done from a couple.  We will
% see later that \emph{Mimic Leads, Flip Back} is a four-person call, done from a wave.


