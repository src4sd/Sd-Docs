\documentclass[12pt]{article}
\usepackage{tex-squares,multicol}
\begin{document}

\title{The Mimic Concept: Centers and Ends}
\author{Sue Curtis \\ Sharon, MA}
\date{\ctablebox{December 2009\\Last Updated May 22, 2023}}
\maketitle

\parskip 12pt
\section{Introduction}

The \emph{Mimic} concept is a way of getting everyone to do a designated
part of the call.
For example, \emph{Mimic Ends} means that everyone does the ends'
part of the call.
\emph{Mimic Centers} is generally the same as Central in cases where
Central is proper, but \emph{Mimic Centers} applies to more calls.
This paper will discuss \emph{Mimic Centers} and \emph{Mimic Ends};
other examples, such as \emph{Mimic Leads}, are discussed elsewhere.
\footnote{Sue Curtis and Bill Ackerman, \emph{The Mimic Concept}, December 2007.}

There are two ways to do \emph{Mimic} calls.  One way is to think about
what you would do if you were in the indicated position
(e.g. centers) and do that part.
The other way to do \emph{Mimic} calls involves making
an adjustment, doing the call, then undoing the adjustment.   
I expect that dancers will eventually learn to do many of the
\emph{Mimic} calls in a single smooth flowing motion, without
making any adjustments, as they now do Central.
However, when dancers are first learning \emph{Mimic}, or when they
later encounter new or unusual examples, they may wish to physically
make the adjustment.  Thus, in this paper, I will describe the
calls in terms of making an adjustment.
You can think of this as similar to the parallelogram adjustment:
we explain the call in terms of an adjustment, but we only physically
make the adjustment on harder calls.

This paper gives a definition of \emph{Mimic} tailored to examples
using Centers and Ends.  The next section provides the definition and
examples.  The last two sections briefly discuss some issues on splitting
the setup and determining the final adjustment.  These issues can
be complex in the full Mimic concept, but tend to be much simpler with
\emph{Mimic Centers} or \emph{Mimic Ends}.

\section{Definition and Examples}

While we have so far described the concept as
\emph{Mimic Centers} or \emph{Mimic Ends}, the full syntax for the concept
is \emph{Mimic $<$position$>$ of $<$formation$>$}, such as
\emph{Mimic Ends of Lines}.
Callers may use the shorthand syntax if there is no ambiguity,
but generally the full syntax is preferred.

As in the Central concept, Mimic calls always involve working in a group 
of four.  It will usually be each box but could be each line or diamond.
The first step in doing Mimic calls is to add phantoms in such a way
that you become the position named, in the formation named.  Here are
some examples:
\begin{itemize}
\item
For \emph{Mimic Centers of Lines}, add phantoms outside
your box of four (sideways) to make a 2x4 setup with the real
people as Centers of Lines.
\item
For \emph{Mimic Centers of Columns}, add phantoms outside
your box of four (in front or behind you) to make a 2x4 setup with the real
people as Centers of Columns.
\item
For \emph{Mimic Centers of an Hourglass}, add phantoms outside
your diamond so that your diamond becomes the center diamond of an Hourglass.
\item
For \emph{Mimic Ends of Lines}, slide out sideways from your box
so as to become the Ends of Lines.
For \emph{Mimic Ends of Columns}, move forward or backward away from
your box so as to become the Ends of Columns.
\end{itemize}

The second step in doing Mimic calls is to do the call in the
newly formed phantom setup.
The final step is to eliminate the phantoms you added and
collapse the setup back to its original size.
This is often done by eliminating phantoms who came into the center
or finished on the outside, but could also be done by merging
real people with adjacent phantoms.

% these seem too trivial to describe here, it's in examples later
% \example{Mimic Centers of Lines}
% \displaytwo
% {\dancer{1}{n}\dancer{2}{s}\cr
%  \dancer{3}{n}\dancer{4}{s}}
% {Before Call}
% {\pdancer{}{x}\dancer{1}{n}\dancer{2}{s}\pdancer{}{x}\cr
%  \pdancer{}{x}\dancer{3}{n}\dancer{4}{s}\pdancer{}{x}}
% {After Adding Phantoms}
% \endexample
% 
% \example{Mimic Centers of Columns}
% \displaytwo
% {\dancer{1}{n}\dancer{2}{s}\cr
%  \dancer{3}{n}\dancer{4}{s}}
% {Before Call}
% {\pdancer{}{x}\pdancer{}{x}\cr
%  \dancer{1}{n}\dancer{2}{s}\cr
%  \dancer{3}{n}\dancer{4}{s}\cr
%  \pdancer{}{x}\pdancer{}{x}}
% {After Adding Phantoms}
% \endexample

The easiest examples for people to learn are those that are similar
to Central.  You may have already been at a dance where someone
tried to apply Central to Disband or Expand the Column.
These applications are considered improper because the dancers who start in
the center finish on the outside.
With Mimic, we have a formal definition that specifies phantoms
are added at the beginning and then removed at the end.
Below is how we would do \emph{Mimic Centers, Disband}.
(The diagram illustrates \emph{Mimic Centers of Lines, Disband}, but
\emph{Mimic Centers of Columns, Disband} produces the same results.)

\example{Mimic Centers, Disband}
\displayfour
{\dancer{1}{n}\dancer{2}{s}\cr
 \dancer{3}{n}\dancer{4}{s}}
{Before Call}
{\pdancer{}{x}\dancer{1}{n}\dancer{2}{s}\pdancer{}{x}\cr
 \pdancer{}{x}\dancer{3}{n}\dancer{4}{s}\pdancer{}{x}}
{After Adding Phantoms}
{\dancer{3}{w}\dancer{1}{w}\cr
 \pdancer{}{x}\pdancer{}{x}\cr
 \pdancer{}{x}\pdancer{}{x}\cr
 \dancer{4}{e}\dancer{2}{e}}
{After Disband}
{\dancer{3}{w}\dancer{1}{w}\cr
 \dancer{4}{e}\dancer{2}{e}}
{After Entire Call}
\endexample

When danced in a single smooth flowing motion (i.e., without adjustments), 
\emph{Mimic Centers, Disband} is
Box Counter Rotate and U-Turn Back away from your partner.

Below are some other examples that would be improper for use with Central.

\example{Mimic Centers of Columns, Strut Right}
\displayfour
{\dancer{1}{s}\dancer{2}{s}\cr
 \dancer{3}{n}\dancer{4}{n}}
{Before Call}
{\pdancer{}{x}\pdancer{}{x}\cr
 \dancer{1}{s}\dancer{2}{s}\cr
 \dancer{3}{n}\dancer{4}{n}\cr
 \pdancer{}{x}\pdancer{}{x}}
{\ctablebox{After Adding\\Phantoms}}
{\dancer{1}{n}\pdancer{}{x}\dancer{2}{n}\pdancer{}{x}
 \pdancer{}{x}\dancer{3}{s}\pdancer{}{x}\dancer{4}{s}}
{After Strut Right}
{\dancer{1}{n}\dancer{2}{n}\dancer{3}{s}\dancer{4}{s}}
{After Entire Call}
\endexample

\example{Mimic Centers of a 1/4 Line, Finish Nuclear Reaction}
\displayfour
{\dancer{1}{n}\dancer{2}{n}\dancer{3}{s}\dancer{4}{s}}
{Before Call}
{\idancer\pdancer{}{x}\pdancer{}{x}\idancer\cr
 \dancer{1}{n}\dancer{2}{n}\dancer{3}{s}\dancer{4}{s}\cr
 \idancer\pdancer{}{x}\pdancer{}{x}\idancer}
{After Adding Phantoms}
{\dancer{4}{w}\dancer{3}{w}\cr
 \pdancer{}{x}\pdancer{}{x}\cr
 \pdancer{}{x}\pdancer{}{x}\cr
 \dancer{2}{e}\dancer{1}{e}}
{\ctablebox{After Finish\\Nuclear Reaction}}
{\dancer{4}{w}\dancer{3}{w}\cr
 \dancer{2}{e}\dancer{1}{e}}
{After Entire Call}
\endexample

Another way that \emph{Mimic Centers} extends the Central concept
is to allow cases where the centers' part is different based on
whether the overall setup is lines or columns.

\example{Mimic Centers of Lines, Drift Apart}
\displayfour
{\dancer{1}{n}\dancer{2}{s}\cr
 \dancer{3}{n}\dancer{4}{s}}
{Before Call}
{\pdancer{}{x}\dancer{1}{n}\dancer{2}{s}\pdancer{}{x}\cr
 \pdancer{}{x}\dancer{3}{n}\dancer{4}{s}\pdancer{}{x}}
{After Adding Phantoms}
{\dancer{4}{n}\pdancer{}{x}\pdancer{}{x}\dancer{3}{s}\cr
 \dancer{2}{n}\pdancer{}{x}\pdancer{}{x}\dancer{1}{s}}
{After Drift Apart}
{\dancer{4}{n}\dancer{3}{s}\cr
 \dancer{2}{n}\dancer{1}{s}}
{After Entire Call}
\endexample

\example{Mimic Centers of Columns, Drift Apart}
\displayfour
{\dancer{1}{n}\dancer{2}{s}\cr
 \dancer{3}{n}\dancer{4}{s}}
{Before Call}
{\pdancer{}{x}\pdancer{}{x}\cr
 \dancer{1}{n}\dancer{2}{s}\cr
 \dancer{3}{n}\dancer{4}{s}\cr
 \pdancer{}{x}\pdancer{}{x}}
{After Adding Phantoms}
{\dancer{1}{s}\dancer{3}{n}\cr
 \pdancer{}{x}\pdancer{}{x}\cr
 \pdancer{}{x}\pdancer{}{x}\cr
 \dancer{2}{s}\dancer{4}{n}}
{After Drift Apart}
{\dancer{1}{s}\dancer{3}{n}\cr
 \dancer{2}{s}\dancer{4}{n}}
{After Entire Call}
\endexample

Mimic Ends is potentially even more useful than Mimic Centers since
we do not already have a concept analogous to Central that specifies the
ends' part.  Below are some examples.

\example{Mimic Ends of Lines, Detour}
\displayfour
{\dancer{1}{n}\dancer{2}{s}\cr
 \dancer{3}{n}\dancer{4}{s}}
{Before Call}
{\dancer{1}{n}\pdancer{}{x}\pdancer{}{x}\dancer{2}{s}\cr
 \dancer{3}{n}\pdancer{}{x}\pdancer{}{x}\dancer{4}{s}}
{After Adding Phantoms}
{\dancer{3}{w}\pdancer{}{x}\pdancer{}{x}\dancer{4}{w}\cr
 \dancer{1}{e}\pdancer{}{x}\pdancer{}{x}\dancer{2}{e}}
{After Detour}
{\dancer{3}{w}\dancer{4}{w}\cr
 \dancer{1}{e}\dancer{2}{e}}
{After Entire Call}
\endexample

This is the same as Reset 1/2, and many dancers could probably do it
by thinking of the ends' part of Detour, and not explicitly adding phantoms.
Below is another example you could probably do without explicitly adding
phantoms, but this one we don't have another name for.

\example{Mimic Ends of Lines, Cast a Shadow}
\displayfour
{\dancer{1}{n}\dancer{2}{s}\cr
 \dancer{3}{n}\dancer{4}{s}}
{Before Call}
{\dancer{1}{n}\pdancer{}{x}\pdancer{}{x}\dancer{2}{s}\cr
 \dancer{3}{n}\pdancer{}{x}\pdancer{}{x}\dancer{4}{s}}
{After Adding Phantoms}
{\dancer{1}{w}\dancer{2}{w}\cr
 \pdancer{}{x}\pdancer{}{x}\cr
 \pdancer{}{x}\pdancer{}{x}\cr
 \dancer{3}{e}\dancer{4}{e}}
{After Cast a Shadow}
{\dancer{1}{w}\dancer{2}{w}\cr
 \dancer{3}{e}\dancer{4}{e}}
{After Entire Call}
\endexample

When done in a single smooth flowing motion, this call is
Reset 1/4 and Cast Off 3/4.

Below are some other examples I like.

\example{Mimic Ends of Lines, Disband}
\displayfour
{\dancer{1}{n}\dancer{2}{s}\cr
 \dancer{3}{n}\dancer{4}{s}}
{Before Call}
{\dancer{1}{n}\pdancer{}{x}\pdancer{}{x}\dancer{2}{s}\cr
 \dancer{3}{n}\pdancer{}{x}\pdancer{}{x}\dancer{4}{s}}
{After Adding Phantoms}
{\pdancer{}{x}\pdancer{}{x}\cr
 \dancer{4}{w}\dancer{2}{w}\cr
 \dancer{3}{e}\dancer{1}{e}\cr
 \pdancer{}{x}\pdancer{}{x}}
{After Disband}
{\dancer{4}{w}\dancer{2}{w}\cr
 \dancer{3}{e}\dancer{1}{e}}
{After Entire Call}
\endexample

When done in a single smooth flowing motion, this call is
Reset 1/4 and 2/3 Recycle.
The adjustment shown in this example illustrates Mimic Ends of Lines,
but you would get the same result for Mimic Ends of Columns.

Mimic Ends can also be used with various lines-facing calls such as
Load the Boat, Square the Bases, and Chisel Thru.  Below is Plan Ahead.

\example{Mimic Ends of Lines, Plan Ahead}
\displayfour
{\dancer{1}{s}\dancer{2}{s}\cr
 \dancer{3}{n}\dancer{4}{n}}
{Before Call}
{\dancer{1}{s}\pdancer{}{x}\pdancer{}{x}\dancer{2}{s}\cr
 \dancer{3}{n}\pdancer{}{x}\pdancer{}{x}\dancer{4}{n}}
{After Adding Phantoms}
{\pdancer{}{x}\dancer{2}{e}\dancer{4}{w}\pdancer{}{x}\cr
 \pdancer{}{x}\dancer{1}{e}\dancer{3}{w}\pdancer{}{x}}
{After Plan Ahead}
{\dancer{2}{e}\dancer{4}{w}\cr
 \dancer{1}{e}\dancer{3}{w}}
{After Entire Call}
\endexample

When done in a single smooth flowing motion, this call is
Box Circulate 1 1/2 and Recycle.

Another set of examples involves 3/4 Tags.  Think of the outsides'
part of Little, Stampede, Rally, or even Plenty.  Below is Counter.

\example{Mimic Outsides of a 3/4 Tag, Counter}
\displayfour
{\dancer{1}{n}\dancer{2}{n}\cr
 \dancer{3}{s}\dancer{4}{s}}
{Before Call}
{\idancer\dancer{1}{n}\dancer{2}{n}\idancer\cr
 \pdancer{}{x}\pdancer{}{x}\pdancer{}{x}\pdancer{}{x}\cr
 \idancer\dancer{3}{s}\dancer{4}{s}\idancer}
{After Adding Phantoms}
{\dancer{1}{s}\pdancer{}{x}\pdancer{}{x}\dancer{2}{n}\cr
 \dancer{3}{s}\pdancer{}{x}\pdancer{}{x}\dancer{4}{n}}
{After Counter}
{\dancer{1}{s}\dancer{2}{n}\cr
 \dancer{3}{s}\dancer{4}{n}}
{After Entire Call}
\endexample

When done smoothly this call is U-Turn Back away from your partner,
Touch 1/2, Step and Fold.

\section{Splitting Setups}

As with the Central concept, Mimic always requires working in a group of 4.
All of the examples illustrated so far were shown with only 4 dancers
and thus avoided the potential issue of deciding which group of 4 to
work in.  From some setups, such as a 1x8 or diamonds,
this is straight-forward, since the setup can only be split one way.
From a 2x4, there are potentially two ways: each box and each 1x4.
However, callers can avoid any ambiguity by using suitable full syntax.
For example, ``Mimic Centers of Lines'', ``Mimic Centers of Columns'',
``Mimic Ends of Lines'', ``Mimic Ends of Columns, and
``Mimic Outsides of a 3/4 Tag'' all require that you work in your
box.
Conversely, ``Mimic Centers of Diamonds'' and
 ``Mimic Center 4 of a Tidal Wave'' always refer to each wave.

There is one more thing worth pointing out about the process of adding
phantoms.  Often, adding the phantoms will cause the real people to work
in Split Phantom Lines, Columns, or Boxes.  This is illustrated below.

\example{Mimic Centers of Lines}
\displaytwo
{\dancer{1}{n}\dancer{2}{s}\dancer{3}{n}\dancer{4}{s}\cr
 \dancer{5}{n}\dancer{6}{s}\dancer{7}{n}\dancer{8}{s}}
{Before Call}
{\pdancer{}{x}\dancer{1}{n}\dancer{2}{s}\pdancer{}{x}
 \pdancer{}{x}\dancer{3}{n}\dancer{4}{s}\pdancer{}{x}\cr
 \pdancer{}{x}\dancer{5}{n}\dancer{6}{s}\pdancer{}{x}
 \pdancer{}{x}\dancer{7}{n}\dancer{8}{s}\pdancer{}{x}}
{After Adding Phantoms}
\endexample

In this example, the dancers in each box add phantoms outside of their box
to form lines.  This essentially forces them into the outside Triple Boxes.
They then effectively do the call in Split Phantom Boxes.

\example{Mimic Ends of Columns}
\displaytwo
{\dancer{1}{n}\dancer{2}{s}\dancer{3}{n}\dancer{4}{s}\cr
 \dancer{5}{n}\dancer{6}{s}\dancer{7}{n}\dancer{8}{s}}
{Before Call}
{\dancer{1}{n}\dancer{2}{s}\dancer{3}{n}\dancer{4}{s}\cr
 \pdancer{}{x}\pdancer{}{x}\pdancer{}{x}\pdancer{}{x}\cr
 \pdancer{}{x}\pdancer{}{x}\pdancer{}{x}\pdancer{}{x}\cr
 \dancer{5}{n}\dancer{6}{s}\dancer{7}{n}\dancer{8}{s}}
{After Adding Phantoms}
\endexample

In this example, the dancers in each box add phantoms inside their box
so as to turn their box into a column.  They then effectively do the
call in Split Phantom Columns.

In either Mimic example, unlike the Split Phantom concepts,
the phantoms are removed at the end.

\section{Collapsing the Setup}

In the examples shown so far, there was only one way to eliminate
phantoms (after finishing the call) and collapse the setup
back to its original size.
This is true of most \emph{Mimic Centers} and \emph{Mimic Ends} calls:
the phantoms will usually end up either in the center or on the end,
and are easy to eliminate.  One minor exception was shown earlier with
Strut Right, but even in this case there was no ambiguity in eliminating
phantoms.

Potentially ambiguous examples may arise if Mimic is used with calls
where some of the real people finish in the center of a 2x4 and others finish
on the end.  In these situations, there may be multiple ways to
eliminate phantoms and collapse the setup.  In the
event of ambiguity, choose the ending that
makes the call a non-shape-changer (i.e. if the call starts in a box,
it ends in a box.)

\example{Mimic Ends of Lines, Scatter Circulate}
\displayfour
{\dancer{1}{n}\dancer{2}{s}\cr
 \dancer{3}{n}\dancer{4}{s}}
{Before Call}
{\dancer{1}{n}\pdancer{}{x}\pdancer{}{x}\dancer{2}{s}\cr
 \dancer{3}{n}\pdancer{}{x}\pdancer{}{x}\dancer{4}{s}}
{After Adding Phantoms}
{\dancer{3}{n}\dancer{1}{s}\pdancer{}{x}\pdancer{}{x}\cr
 \pdancer{}{x}\pdancer{}{x}\dancer{4}{n}\dancer{2}{s}}
{After Scatter Circulate}
{\dancer{3}{n}\dancer{1}{s}\cr
 \dancer{4}{n}\dancer{2}{s}}
{Ending Setup}
\endexample

After adding phantoms and doing the Scatter Circulate, you could imagine
eliminating phantoms by adjusting the real people to a box (moving
sideways) or to a line (moving forward or back).  The rule is
that the box ending is preferred.  This example turns
out to be equivalent to Box Circulate.

\section{Conclusions}

The purpose of the \emph{Mimic} concept is to provide a way of
doing other dancers' parts of the call, such as having everyone
do the leads' part or everyone doing the ends' part.
This paper illustrates \emph{Mimic Centers} and \emph{Mimic Ends}.
\emph{Mimic Centers} extends the Central concept by allowing calls where
the centers do not stay in the center (e.g. Disband, Strut Right) or
calls where the centers have different parts based on whether the setup
is lines or columns (e.g. Drift Apart).  \emph{Mimic Ends} essentially
creates a new concept that is analogous to Central but involves doing
the ends' part instead of the centers' part.

In my experience, \emph{Mimic Centers} and \emph{Mimic Ends} are
easier to learn than the more generalized \emph{Mimic} concept.
This happens for several reasons.  First, dancers at c3 and above
are already familiar with the Central concept and are already accustomed to
doing the centers' part of calls in their own box.  Second, these
examples tend to have dancers working together in a way that it is
obvious how to remove the phantoms at the end.

The impact of \emph{Mimic} is to create many more four-person calls
than we currently have.  The only other concepts I can think of
that create four-person calls from eight-person calls are Central
and Single.  However, these concepts are much more restrictive and
only apply to certain types of calls.  \emph{Mimic} is more general and
presumably generates many more new calls.

\section{Acknowledgments}

I would like to thank Bill Ackerman, Andy Latto, Eric Brosius,
and Will and Mary Leland
for helpful suggestions and comments on the original \emph{Mimic} concept.
I would also like to thank the many dancers at C-Gulls, Lake Shore,
and Berkshires who tolerated my experiments with this concept and
provided useful feedback.

\newpage
\section*{Practice Sequences}
\bigskip
\sequence{Mostly Centers}
HEADS: touch 1/4\\
quick step\\
walk out to a wave\\
counter rotate\\
MIMIC CENTERS OF LINES, disband\\
circulate\\
MIMIC ENDS OF LINES, disband\\
tally ho\\
MIMIC CENTERS OF LINES, sets in motion\\
plan ahead\\
double pass thru\\
MIMIC CENTERS OF COLUMNS, expand the column\\
touch by 1/4 and 1/4\\
scatter circulate\\
MIMIC CENTERS OF LINES, the gamut\\
counter rotate\\
left allemande  (3/4 promenade)\\
\endsequence

\sequence{Mostly Ends}
HEADS: turn thru\\
wave the beaus\\
2/3 recycle\\
MIMIC POINTS OF DIAMONDS, strip the diamonds\\
along\\
MIMIC ENDS OF LINES, acey deucey\\
step and fold\\
MIMIC CENTERS OF LINES, drift apart\\
circulate\\
invert the column 1/2\\
MIMIC ENDS OF LINES, detour\\
peel and trail\\
flip the line\\
boomerang\\
right and left grand  (7/8 promenade)\\
\endsequence

\end{document}
